% mnras_template.tex 
%
% LaTeX template for creating an MNRAS paper
%
% v3.0 released 14 May 2015
% (version numbers match those of mnras.cls)
%
% Copyright (C) Royal Astronomical Society 2015
% Authors:
% Keith T. Smith (Royal Astronomical Society)

% Change log
%
% v3.2 July 2023
%	Updated guidance on use of amssymb package
% v3.0 May 2015
%    Renamed to match the new package name
%    Version number matches mnras.cls
%    A few minor tweaks to wording
% v1.0 September 2013
%    Beta testing only - never publicly released
%    First version: a simple (ish) template for creating an MNRAS paper

%%%%%%%%%%%%%%%%%%%%%%%%%%%%%%%%%%%%%%%%%%%%%%%%%%
% Basic setup. Most papers should leave these options alone.
\documentclass[fleqn,usenatbib]{mnras}

% MNRAS is set in Times font. If you don't have this installed (most LaTeX
% installations will be fine) or prefer the old Computer Modern fonts, comment
% out the following line

%\usepackage{newtxtext,newtxmath}
% Depending on your LaTeX fonts installation, you might get better results with one of these:
%\usepackage{mathptmx}
%\usepackage{txfonts}
% Use vector fonts, so it zooms properly in on-screen viewing software
% Don't change these lines unless you know what you are doing
\usepackage[T1]{fontenc}

% Allow "Thomas van Noord" and "Simon de Laguarde" and alike to be sorted by "N" and "L" etc. in the bibliography.
% Write the name in the bibliography as "\VAN{Noord}{Van}{van} Noord, Thomas"
\DeclareRobustCommand{\VAN}[3]{#2}
\let\VANthebibliography\thebibliography
\def\thebibliography{\DeclareRobustCommand{\VAN}[3]{##3}\VANthebibliography}


%%%%% AUTHORS - PLACE YOUR OWN PACKAGES HERE %%%%%

% Only include extra packages if you really need them. Avoid using amssymb if newtxmath is enabled, as these packages can cause conflicts. newtxmatch covers the same math symbols while producing a consistent Times New Roman font. Common packages are:
\usepackage{graphicx}	% Including figure files
\usepackage{amsmath}	% Advanced maths commands

%%%%%%%%%%%%%%%%%%%%%%%%%%%%%%%%%%%%%%%%%%%%%%%%%%

%%%%% AUTHORS - PLACE YOUR OWN COMMANDS HERE %%%%%

% Please keep new commands to a minimum, and use \newcommand not \def to avoid
% overwriting existing commands. Example:
%\newcommand{\pcm}{\,cm$^{-2}$}	% per cm-squared

%%%%%%%%%%%%%%%%%%%%%%%%%%%%%%%%%%%%%%%%%%%%%%%%%%

%%%%%%%%%%%%%%%%%%% TITLE PAGE %%%%%%%%%%%%%%%%%%%

% Title of the paper, and the short title which is used in the headers.
% Keep the title short and informative.
\title[Extragalactic Astronomy]{Balogh vs Dressler: Spectral classification}

% The list of authors, and the short list which is used in the headers.
% If you need two or more lines of authors, add an extra line using \newauthor
\author[Mary Verdugo]{
Mary Verdugo,$^{1}$\thanks{E-mail: mary.verdugo@userena.cl}
\\
% List of institutions
$^{1}$Estudiante de Magister en Astronomia, Universidad de La Serena\\
}

% These dates will be filled out by the publisher
\date{Mon, Dic 4}

% Enter the current year, for the copyright statements etc.
\pubyear{2023}

% Don't change these lines
\begin{document}
\label{firstpage}
\pagerange{\pageref{firstpage}--\pageref{lastpage}}
\maketitle

% Abstract of the paper


% Select between one and six entries from the list of approved keywords.
% Don't make up new ones.


%%%%%%%%%%%%%%%%%%%%%%%%%%%%%%%%%%%%%%%%%%%%%%%% %%

%%%%%%%%%%%%%%%%% BODY OF PAPER %%%%%%%%%%%%%%%%%%

\section{Spectral classification}

Follow the methods obtains from \citet[]{Balogh} with the data from
\citet[]{Dressler}, the aim of this essay is replicate the same figures
9 and 11 from the first paper mentioned with the data of the second, to understand how the classification of Balogh is better in fuction to separete between star forming and passive galaxies.
\\

\section{Methods}
One of the first thing that we do to understand the comparison between the two jobs, is to review the meta-data of the data to homogenize the tables. In the case of \citet{Dressler} we have: \textit{the EW of the $H\delta$  is negative when the line present \textbf{emission}} (see figure \ref{dresslertable}). In comparison, the description about this two line features in \citet{Balogh} is: \textit{... When the EW [O II] index is positive the line is in emission, while the EW $H\delta$ index is positive when the line is in \textbf{absorption}.}. This implies that the data for $H\delta$ from the \citet{Dressler} catalogue is the same in both catalogues,and [O II] needs to be multiplied by $-1$ to homogenize the data. Doing that, we replicate the fig. 9 and fig. 11 from the paper of \citet{Balogh} (see figure \ref{figsbaloghs}).
\\

Using a \texttt{python} libraries, we made the sames figures (see figure \ref{plotmio1} and figure \ref{plotmio2}) 

\begin{figure}
	\includegraphics[width=\columnwidth]{/home/patito/Documents/egdatos/tableparameters.png}
    \caption{Capture of the table 5 from \citet{Dressler}}
    \label{dresslertable}
\end{figure}

\begin{figure}
	\includegraphics[width=\columnwidth]{/home/patito/Documents/egdatos/pdf/mnrastemplate/imagecomparison.png}
    \caption{figures 9 and 11 from the paper of \citet{Balogh}}
    \label{figsbaloghs}
\end{figure}

\begin{figure}
	\includegraphics[width=\columnwidth]{//home/patito/Documents/egdatos/pdf/mnrastemplate/plotmio1.png}
	\caption{Figure made to replicate the first panel of the figure \ref{figsbaloghs}, this show the relation between the EW of $O[II]$ and EW of $H\delta$}
    \label{plotmio1}
\end{figure}

\begin{figure}
	\includegraphics[width=\columnwidth]{//home/patito/Documents/egdatos/pdf/mnrastemplate/plotmio2.png}
	\caption{Figure made to replicate the second panel of the figure \ref{figsbaloghs}, this show the relation between the EW of $O[II]$ and EW of $H\delta$}
    \label{plotmio2}
\end{figure}

\begin{figure}
	\includegraphics[width=\columnwidth]{//home/patito/Documents/egdatos/pdf/mnrastemplate/creacionpropia.png}
	\caption{This figure show the relation between EW of $O[II]$ vs D4000}
    \label{plotcreado}
\end{figure}


\begin{figure}
	\includegraphics[width=\columnwidth]{///home/patito/Documents/egdatos/pdf/mnrastemplate/morfologicohdeltaoii.png}
	\caption{Figure to replicate the second panel of the figure \ref{figsbaloghs}, were replacing the morphology of the galaxies in function to have the sames of \citet{Balogh}}.
	\label{plotmorph1}
\end{figure}

\begin{figure}
	\includegraphics[width=\columnwidth]{//home/patito/Documents/egdatos/pdf/mnrastemplate/morfologicoHdeltad4000.png}
	\caption{Figure made to replicate the first panel of the figure \ref{figsbaloghs}, were replacing the morphology of the galaxies in function to have the sames of \citet{Balogh}}
	\label{plotmorph2}
\end{figure}

\begin{figure}
	\includegraphics[width=\columnwidth]{//home/patito/Documents/egdatos/pdf/mnrastemplate/membergalaxies.png}
	\caption{Histogram for the count of galaxies per cluster}
	\label{histogram}
\end{figure}




\section{Analysis}
We made the sames graphics from \citet{Balogh} ussing the data provided by \citet{Dressler}, but with the aim to do the same morphological classification, we depure the data, excluding the galaxies from the \textit{field} and unified the morphological classification from \citet{Dressler}, i.e. we use the column of \textit{MType} in function to obtain \textit{E/S0, Sab, Sbcd, Irr/SB}, see figure \ref{plotmorph1} and \ref{plotmorph2}. And also, we made the plots separate by cluster. \\

In two cases, we can see a poorly stadistic. In order to confirm, we made the count of number of galaxies per cluster in the figure \ref{histogram}. It is important understand the context, this two papers was the first in the area of the spectral classification, but the data was be used carefully. \\

\subsection{H$\delta$ and the stellar populations}
H$\delta$ is an spectral line of absorption related with the presence of the stars A and F




%%%%%%%%%%%%%%%%%%%% REFERENCES %%%%%%%%%%%%%%%%%%

% The best way to enter references is to use BibTeX:

\bibliographystyle{mnras}
\bibliography{example} % if your bibtex file is called example.bib


% Alternatively you could enter them by hand, like this:
% This method is tedious and prone to error if you have lots of references
%\begin{thebibliography}{99}
%\bibitem[\protect\citeauthoryear{Author}{2012}]{Author2012}
%Author A.~N., 2013, Journal of Improbable Astronomy, 1, 1
%\bibitem[\protect\citeauthoryear{Others}{2013}]{Others2013}
%Others S., 2012, Journal of Interesting Stuff, 17, 198
%\end{thebibliography}

%%%%%%%%%%%%%%%%%%%%%%%%%%%%%%%%%%%%%%%%%%%%%%%%%%

%%%%%%%%%%%%%%%%% APPENDICES %%%%%%%%%%%%%%%%%%%%%

% Don't change these lines
\bsp	% typesetting comment
\label{lastpage}
\end{document}

% End of mnras_template.tex
